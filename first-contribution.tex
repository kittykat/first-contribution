% $Header: /Users/joseph/Documents/LaTeX/beamer/solutions/conference-talks/conference-ornate-20min.en.tex,v 90e850259b8b 2007/01/28 20:48:30 tantau $

\documentclass{beamer}

% This file is a solution template for:

% - Talk at a conference/colloquium.
% - Talk length is about 20min.
% - Style is ornate.



% Copyright 2004 by Till Tantau <tantau@users.sourceforge.net>.
%
% In principle, this file can be redistributed and/or modified under
% the terms of the GNU Public License, version 2.
%
% However, this file is supposed to be a template to be modified
% for your own needs. For this reason, if you use this file as a
% template and not specifically distribute it as part of a another
% package/program, I grant the extra permission to freely copy and
% modify this file as you see fit and even to delete this copyright
% notice. 


\mode<presentation>
{
  \usetheme{Frankfurt}
  % or ...

  \setbeamercovered{transparent}
  % or whatever (possibly just delete it)
}


\usepackage[english]{babel}
% or whatever

\usepackage[latin1]{inputenc}
% or whatever

\usepackage{times}
\usepackage[T1]{fontenc}
% Or whatever. Note that the encoding and the font should match. If T1
% does not look nice, try deleting the line with the fontenc.


\title%[Short Paper Title] (optional, use only with long paper titles)
{First steps towards contributing}

%\subtitle
%{Include Only If Paper Has a Subtitle}

\author
{Ekaterina Gerasimova, \texttt{kittykat3756@gmail.com} \and \\Sindhu S, \texttt{email@example.com}}
% - Give the names in the same order as the appear in the paper.
% - Use the \inst{?} command only if the authors have different
%   affiliation.

%\institute[Universities of Somewhere and Elsewhere] (optional, but mostly needed)
%{
%  \inst{1}%
%  Department of Computer Science\\
%  University of Somewhere
%  \and
%  \inst{2}%
%  Department of Theoretical Philosophy\\
%  University of Elsewhere}
% - Use the \inst command only if there are several affiliations.
% - Keep it simple, no one is interested in your street address.

\date%[GUADEC 2013] % (optional, should be abbreviation of conference name)
{GUADEC 2013}
% - Either use conference name or its abbreviation.
% - Not really informative to the audience, more for people (including
%   yourself) who are reading the slides online

%\subject{Theoretical Computer Science}
% This is only inserted into the PDF information catalog. Can be left
% out. 



% If you have a file called "university-logo-filename.xxx", where xxx
% is a graphic format that can be processed by latex or pdflatex,
% resp., then you can add a logo as follows:

\pgfdeclareimage[height=1cm]{university-logo}{guadec.pdf}
\logo{\pgfuseimage{university-logo}}



% Delete this, if you do not want the table of contents to pop up at
% the beginning of each subsection:
%\AtBeginSubsection[]
%{
%  \begin{frame}<beamer>{Outline}
%    \tableofcontents[currentsection,currentsubsection]
%  \end{frame}
%}


% If you wish to uncover everything in a step-wise fashion, uncomment
% the following command: 

%\beamerdefaultoverlayspecification{<+->}


\begin{document}

\begin{frame}
  \titlepage
\end{frame}

%\begin{frame}{Outline}
%  \tableofcontents
  % You might wish to add the option [pausesections]
%\end{frame}


% Structuring a talk is a difficult task and the following structure
% may not be suitable. Here are some rules that apply for this
% solution: 

% - Exactly two or three sections (other than the summary).
% - At *most* three subsections per section.
% - Talk about 30s to 2min per frame. So there should be between about
%   15 and 30 frames, all told.

% - A conference audience is likely to know very little of what you
%   are going to talk about. So *simplify*!
% - In a 20min talk, getting the main ideas across is hard
%   enough. Leave out details, even if it means being less precise than
%   you think necessary.
% - If you omit details that are vital to the proof/implementation,
%   just say so once. Everybody will be happy with that.

\section{How does it work?}

\subsection{Why are you here?}

\begin{frame}{Why are you here?}%{Subtitles are optional.}
  \pgfdeclareimage[height=4.5cm]{bad-flowchart}{bad-flowchart.pdf}
  \center{\pgfuseimage{bad-flowchart}}
  % If you think the first step towards contributing is sending in a patch, you're probably wrong.
\end{frame}

\begin{frame}{It's all about the people}
  \begin{itemize}
  \item
    Talk to people\ldots
    \begin{itemize}
    \item
      IRC
    \item
      Mailing lists
    \item
      Here and now
    \end{itemize}
  \item
    Use available resources
  \item
    Ask for help
  \end{itemize}
\end{frame}

\begin{frame}{Code of conduct}
  \begin{itemize}
  \item
    Assume people mean well
  \item
    Try to be concise
  \item
    Be patient and generous
  \item
    Be respectful and considerate
  \end{itemize}
\end{frame}

\begin{frame}{Workflow}
  \begin{itemize}
  \item
    Bugzilla?
  \item
  \item
  \item
  \end{itemize}
\end{frame}

\begin{frame}{Followup}
  \begin{itemize}
  \item
    Be patient
  \item
    Follow up with the reviewer
  \item
    Follow up on the review
  \end{itemize}
\end{frame}

\begin{frame}{Make good contributions}
  \begin{itemize}
  \item
    
  \item
  \end{itemize}
\end{frame}

\begin{frame}{How things happen}
\end{frame}

\begin{frame}{Experience}
  \begin{itemize}
  \item
    The more you contribute, the more you communicate with people
  \item
    The more you communicate, the more you become part of the community
  \item
    Once you are part of the community, you can become a Foundation member
  \end{itemize} 
\end{frame}


\section{The reality}

\subsection{need title}

\begin{frame}{Sindhu's slide 1}
\end{frame}

\begin{frame}{Sindhu's slide 2}
\end{frame}


\section*{Summary}

\begin{frame}{Summary}

  % Keep the summary *very short*.
  \begin{itemize}
  \item
    The \alert{first main message} of your talk in one or two lines.
  \item
    The \alert{second main message} of your talk in one or two lines.
  \item
    Perhaps a \alert{third message}, but not more than that.
  \end{itemize}
  
  % The following outlook is optional.
  \vskip0pt plus.5fill
  \begin{itemize}
  \item
    Outlook
    \begin{itemize}
    \item
      Something you haven't solved.
    \item
      Something else you haven't solved.
    \end{itemize}
  \end{itemize}
\end{frame}



% All of the following is optional and typically not needed. 
\appendix
\section<presentation>*{\appendixname}
\subsection<presentation>*{Resources}

\begin{frame}[allowframebreaks]
  \frametitle<presentation>{Resources}
    
  \begin{thebibliography}{10}

  \bibitem{code}
    Code of conduct:
    \newblock https://wiki.gnome.org/CodeOfConduct/

  \end{thebibliography}
\end{frame}

\end{document}
